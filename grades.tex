\documentclass[a4paper,12pt]{article}
\usepackage{cmap}
\usepackage[utf8]{inputenc}
\usepackage[warn]{mathtext}
\usepackage{epsf,amsmath,amsfonts,amssymb,amsbsy}
\usepackage[mathscr]{eucal}
\usepackage[english, russian]{babel}
\usepackage[left=1cm,right=2cm,top=2cm,bottom=2cm]{geometry}
\usepackage{graphicx}
\usepackage{indentfirst}
\DeclareGraphicsExtensions{.pdf,.png,.jpg}
\usepackage{pgfplots}
\begin{document}

\newpage

\section{Аннотация}
В данной статье исследуется зависимость успеваемости обучающихся и их среднего балла от различных признаков таких, как курение, частота прогулок на свежем воздухе, занятия спортом и т.д.
Проведён опрос среди обучающихся, в котором приняло участие 230 студентов различных курсов МФТИ. В работе представлены результаты проведенного опроса. Обучена модель для получения прогнозируемого среднего балла.
\section{Сбор данных}
Мы распространили гугл-форму среди студентов Физтеха 1-4 курсов различных Физтех-школ. На неё откликнулось 230 студентов,  что довольно много, учитывая тот факт, что в МФТИ на баклавриате обучается около 5000 человек. 
\begin{figure}[h!]
    \centering	
    \includegraphics[width=1.05\textwidth]{1.jpg}
    \label{1_1}
\end{figure}

\section{Результаты гугл-формы}
Мы выгрузили ответы на гугл-форму в форме csv и обработали их с помощью написанного кода на Python. 

Было полученно распределение опрошенных по некоторым признакам. Некоторые оказались достаточно неожиданными. К примеру, удивительно, но приятно, что курят только $15,7\%$ людей. Также, учитывая, что занятия по физической культуре являются обязательными на Физтехе, внезапным было получить, что $23\%$ обучающихся совсем не занимаются спортом. Неожиданными фактами являются, что около $83\%$ сидят в социальных сетях больше 2 часов, а также, что $57,4 \%$ студентов имеют золотую медаль. Но самым интересным оказалось, что $1/3$ физтехов списывают почти все домашние задания. 

С более подробными результатами можно ознакомится из графиков, представленных ниже.
\begin{figure}[h!]
\begin{tabular}{cc}
\includegraphics[width=0.54\textwidth]{2.jpg}
&
\includegraphics[width=0.54\textwidth]{3.jpg}
\end{tabular}
\end{figure}
\newpage

\begin{figure}[h!]
\begin{tabular}{cc}
\includegraphics[width=0.54\textwidth]{4.jpg}
&
\includegraphics[width=0.54\textwidth]{5.jpg}
\end{tabular}
\end{figure}

\begin{figure}[h!]\centering
\begin{tabular}{cc}
\includegraphics[width=0.54\textwidth]{6.jpg}
&
\includegraphics[width=0.54\textwidth]{7.jpg}
\end{tabular}
\end{figure}

\begin{figure}[h!]\centering
\begin{tabular}{cc}
\includegraphics[width=0.54\textwidth]{8.jpg}
&
\includegraphics[width=0.54\textwidth]{9.jpg}
\end{tabular}
\end{figure}

\begin{figure}[h!]\centering
\begin{tabular}{cc}
\includegraphics[width=0.54\textwidth]{10.jpg}
&
\includegraphics[width=0.54\textwidth]{11.jpg}
\end{tabular}
\end{figure}

\begin{figure}[h!]\centering
    \includegraphics[width=0.8\textwidth]{12.jpg}
    \label{1_1}
\end{figure}
\newpage

\newpage


\section{Зависимость среднего балла от различных факторов}
Далее мы непосредственно приступили к анализу влияния различных факторов на средний балл и изучили распределения по различным выборкам.
\subsection{Влияние пола на средний балл}
Оказалось, что среднее значение среднего балла у женщин оказалось 7.24, что на 0.28 меньше чем у мужчин.
\begin{figure}[h!]\centering
    \includegraphics[width=0.43\textwidth]{13.jpg}
    \label{1_1}
\end{figure}


\subsection{Интрове́рсия}
Априори мы предполагали, что средний балл будет выше у экстравертов, так как они более коммуникабельны, чаще задают вопросы и обращаются за помощью в сложных ситуациях. Но апостериори получилось, что наиболшее среднее значение среднего балла оказалось у интровертов.  
\begin{figure}[h!]\centering
    \includegraphics[width=0.5471\textwidth]{14.jpg}
    \label{1_1}
\end{figure}

\subsection{Алкоголь и курение}
По результатам опроса выяснилось, что курение очень влияет на значение среднего балла. А вот алкоголь не так сильно снижает его. Мы пришли к выводу, что такие результаты связаны с получившейся выборкой - около 50$\%$ студентов физтеха пьют. Наибольший средний балл оказался у людей, которые не курят, и составил $7.52$.
\begin{figure}[h!]\centering
    \includegraphics[width=1\textwidth]{image.png}
    \label{1_1}
\end{figure}
\newpage
\subsection{Компьютерные игры}
Полагаясь на данные из опроса мы получили, что люди, которые совсем не играют в компьютерные игры, имеет выше средний балл. Скорее всего это связано с тем, что игры занимают достаточно много времени.
\begin{figure}[h!]\centering
    \includegraphics[width=0.78\textwidth]{16.jpg}
    \label{1_1}
\end{figure}

\subsection{Спорт}
Было ожидаемо, что студенты, которые тратят много времени на занятия спортом, имеют средний балл низкий. Также априори мы предполагали, что он будет выше у людей, которые занимаются спортом немного, т.к. большиство исследований показывают, что физическая нагрузка увеличивает мозговую активность. Однако апостериори оказалось, что наиболшее среднее значение оказалось у людей, совсем не занимающихся спортом, и составило 7.61.
\begin{figure}[h!]\centering
    \includegraphics[width=0.65\textwidth]{18.jpg}
    \label{1_1}
\end{figure}
\newpage

\subsection{Влияние прогулок на свежем воздухе на средний балл}
Сильное влияние на среднеее значение среднего балла оказывают частые прогулки на свежем воздухе. Как видно из диаграммы ниже, оно на 0.15 балла выше.
\begin{figure}[h!]\centering
    \includegraphics[width=0.41\textwidth]{17.jpg}
    \label{1_1}
\end{figure}


\subsection{Соцсети}
Оказалось, что значение среднего балла у людей, которые проводят в социальных сетях более 2 часов в день, на 0.19 выше, чем у студентов, которые сидят в них меньше. Для авторов этот результат оказался неожиданным.
\begin{figure}[h!]\centering
    \includegraphics[width=0.45\textwidth]{19.jpg}
    \label{1_1}
\end{figure}
\newpage
\subsection{Порядок на рабочем месте}
По результатам, полученным из гугл-формы, оказалось, что соблюдение порядка на рабочем месте практически никак не влияет на средний балл.
\begin{figure}[h!]\centering
    \includegraphics[width=0.45\textwidth]{20.jpg}
    \label{1_1}
\end{figure}

\subsection{Золотая медаль}
Полагаясь на данные из опроса мы получили, что у студентов, имеющих золотую медаль, среднее значение балла выше на 0.54, а именно 7.67. Возможно это связанно с так называемом "синдромом отличника".
\begin{figure}[h!]\centering
    \includegraphics[width=0.45\textwidth]{21.jpg}
    \label{1_1}
\end{figure}
\newpage
\subsection{Списывание домашнего задание}
Самую большую разницу можно заметить между физтехами, которые списываю практически все домашние задания, и теми, кто почти полностью делают всё сами. 
\begin{figure}[h!]\centering
    \includegraphics[width=0.7\textwidth]{22.jpg}
    \label{1_1}
\end{figure}

\section{Интересные зависимости}
Также мы решили посмотреть зависимости, которые не связаны со средним баллом, но достаточно любопытные.
\subsection{Распитие алкоголя старшими и младшими курсами}
Нам захотелось посмотреть, насколько к старшим курсам начинают пить алкоголь больше, и сравнили процент пьющих студентов 1-2 курсов с процентом пьщих физтехов 3-4 курсов. Оказалось, что прибавка составляет около $5\%$.
\begin{figure}[h!]\centering
    \includegraphics[width=0.95\textwidth]{23.jpg}
    \label{1_1}
\end{figure}
\newpage
\subsection{Играют в компютерные игры}
Далее мы решили посмотреть насколько представители разных гендеров увлечены компьютерными играми. Оказалось, что и мужчины, и женщины играют приблизительно одинаково.
\begin{figure}[h!]\centering
    \includegraphics[width=0.95\textwidth]{24.jpg}
    \label{1_1}
\end{figure}

\subsection{Соблюдение порядка парнями и девушками}
Также неожиданным оказалось, что мужчины чаще соблюдают порядок на рабочем месте, чем женщины.
\begin{figure}[h!]\centering
    \includegraphics[width=0.92\textwidth]{25.jpg}
    \label{1_1}
\end{figure}

\section{Матрица корреляций}
Корреляционная матрица – это квадратная таблица, заголовками строк и столбцов которой являются обрабатываемые переменные, а на пересечении строк и столбцов выводятся коэффициенты корреляции для соответствующей пары признаков.

Из неё мы можем увидеть, что:

$\cdot$ Наибольшую (по модулю) корреляцию дают способ выполнения домашнего задания и наличие привычки курения.

$\cdot$ Наименьшую (по модулю) корреляцию дают соблюдение порядка на рабочем месте и увлечение компьютернвми играми.

\begin{figure}[h!]\centering
    \includegraphics[width=1.07\textwidth]{26.jpg}
    \label{1_1}
\end{figure}

\section{Вывод}
Были выявлены зависимости между образом жизни студентов и их средним баллом. Также была построена матрица корреляций и обучена модель. Данных не достаточно, чтобы точно предсказывать средний балл. Для решения данной проблемы возможно собрать больше информации о студентах и увеличить выборки.

В заключение, авторы считают необходимым добавить, что, для того чтобы увеличить успеваемость, можно не только больше учиться, но и поменять некоторые привычки. Например, чаще гулять на свежем воздухе, бросить курить. Это увеличит не только успеваемость, но и качество жизни в целом.
\end{document}

